\documentclass[titlepage]{article}
\usepackage[czech]{babel}
\usepackage[utf8]{inputenc} % set input encoding (not needed with XeLaTeX)
\usepackage{verbatim}
\usepackage[parfill]{parskip}
\usepackage[T1]{fontenc}
\usepackage{hyperref}

\renewcommand\thesection{Fáze \arabic{section}}
\renewcommand\thesubsection{\arabic{section}.\arabic{subsection}}

\title{Dokumentace k semestrální práci předmětu BI-GIT}
\author{Milan Vu}

\begin{document}
\maketitle



\section{Slití repozitářů}
\subsection{Klon repozitáře}
\ldots nepotřebuje další komentář \ldots


\subsection{Klon SVN repozitáře}
\ldots nepotřebuje další komentář \ldots

\subsection{Struktura repozitářů a postup slití}
Zjistil jsem, že se v 1. repozitáři nachází 28 commitů a ve 2. SVN repozitáři 29 commitů.
Také jsem zaznamenal, že jsou poslední 2 commity 1. repozitáře, a poslední 3 commity 2. repozitáře odlišné.
Pomohlo mi to pochopit, proč by se po slití mělo v repozitáři (resp. větvi master) nacházet 31 commitů.

Byly použity příkazy:

\begin{verbatim}
git rev-list --all --count
git remote add svn <svn repo path>
git fetch --all
git log --oneline --graph --all
\end{verbatim}

Přepínač \texttt{--all} mi ukáže log všech commitů, na které ukazuje nějaká reference (\texttt{svn/master}) a ne jen \texttt{master} větev.

\subsection{Rebase}
Byl použit příkaz:

\begin{verbatim}
git log --oneline
\end{verbatim}

pro zjištění adresy větve \texttt{master} a:

\begin{verbatim}
git rebase --onto svn/master 6493a7a master
\end{verbatim}

který přesune větev \texttt{master} za větev \texttt{svn/master}.
Poté jsem opravil konflikty nejdříve v souboru \texttt{index.html} a poté i v \texttt{js/edgebox.js}.
Na závěr jsem pomocí:

\begin{verbatim}
git rebase --continue
\end{verbatim}

dokončil rebase.

\subsection{Smazání reference na SVN}
Byl použit příkaz:
\begin{verbatim}
git remote remove svn
\end{verbatim}

\subsection{Návrh \texttt{.gitignore}}
Byl vytvořen nový soubor \texttt{.gitignore}:

\begin{verbatim*}
echo "*~" >> .gitignore
\end{verbatim*}

Tento soubor zajišťuje, že bude git ignorovat nové soubory, jejichž jméno končí vlnovkou a které se nachází kdekoliv v repositáři.
Bude fungovat pouze v těch revizích, kde bude tento commit jejich rodičem.

\subsection{Commit pod jiným uživatelem}
Byly použity příkazy:

\begin{verbatim}
git add .gitignore
git commit --author="Petr <petr@edgebox.net>"
\end{verbatim}

Přepínač \texttt{--author} je mi znám z přednášek.

\subsection{Pročištění historie od ignorovaných souborů}
Byl použit příkaz:

\begin{verbatim}
git filter-branch --force --tree-filter 'git rm -rf --ignore-unmatch *~ 'HEAD
\end{verbatim}

který projde všechny commity a pro každý commit smaže pomocné soubory.
Po třetím volání už \texttt{refs/heads/master} zůstaly nezměneny, tudíž v žádném commitu už zaručeně nebyly pomocné soubory.

Příkaz jsem našel na \href{https://stackoverflow.com/questions/45732458/ignore-files-committed-to-git-and-also-remove-them-from-history}{stackoverflow}.

\subsection{Pročištění referencí}

Protože jsem volal minulý příkaz s atributem \texttt{--force}, \texttt{original/*} reference byly vyčištěny.

\subsection{Odevzdání fáze 1}
Byly použity příkazy:
\begin{verbatim}
git checkout -b phase1
git push --set-upstream --force origin phase1
\end{verbatim}

Přepínač \texttt{--force} proto, protože to nebyl první pokus.

\section{Čištění repozitáře}
\subsection{FIX of FIX}
\begin{verbatim}
git checkout master
git log --oneline
git rebase -i 8ae3296
\end{verbatim}

Nejdříve jsem se vrátil na větev \texttt{master} pomocí \texttt{git checkout master}.
Poté jsem pomocí \texttt{git log --oneline} vypsal historii commitů a udělal interaktivní rebase nad commitem, který byl před prvním FIX commitem.
Následně jsem změnil pick na squash u celkem 6 commitů.


\subsection{Windows konce řádků}
Byly použity příkazy:
\begin{verbatim}
git filter-branch --force --tree-filter '
sed -i "" 's/^M$//g' index.html'
--tag-name-filter cat -- --all HEAD
\end{verbatim}

Sed příkaz jsem našel na \href{https://stackoverflow.com/questions/2613800/how-to-convert-dos-windows-newline-crlf-to-unix-newline-lf-in-a-bash-script}{stackoverflow}.
Pak už jsem jen dosadil script do kanónu k smazání znaků ze všech commitů.

\underline{Pozn.:} Z nějakého důvodu se změnily commity i před prvním výskytem CRLF a tak jsem musel použít přepínač pro zachování tagů.


\subsection{Odstranění řádků \uv{git-svn-id} z commit messages}
Byly použity příkazy:

\begin{verbatim}
git filter-branch -f --msg-filter '
sed -e "/git-svn-id:/d" -e "/Former-commit-id:/d"'
--tag-name-filter cat -- --all
\end{verbatim}

Opět jsem se inspiroval na \href{https://stackoverflow.com/questions/16092509/how-to-remove-svn-url-from-commit-messages?utm_medium=organic&utm_source=google_rich_qa&utm_campaign=google_rich_qa}{stackoverflow}, ale přidal jsem tam tagy, aby se zachovaly reference.


\subsection{Opravy jmen a adres autorů}
Na opravu jmen a adres autorů jsem použil script:

\begin{verbatim}
git filter-branch --env-filter '
OLD_NAME="petr"
CORRECT_NAME="Petr"
CORRECT_EMAIL="petr@edgebox.net"
if [ "$GIT_COMMITTER_NAME" = "$OLD_NAME" ]
then
export GIT_COMMITTER_NAME="$CORRECT_NAME"
export GIT_COMMITTER_EMAIL="$CORRECT_EMAIL"
fi
if [ "$GIT_AUTHOR_NAME" = "$OLD_NAME" ]
then
export GIT_AUTHOR_NAME="$CORRECT_NAME"
export GIT_AUTHOR_EMAIL="$CORRECT_EMAIL"
fi
' --force --tag-name-filter cat -- --branches --tags --all
\end{verbatim}

který jsem našel \href{https://help.github.com/articles/changing-author-info/}{zde}.

Jen jsem si musel dohledat pomocí \texttt{git log}, která jména jsou špatně.


\subsection{Odevzdání fáze 2}
Byly použity příkazy:

\begin{verbatim}
git checkout -b phase2
git push --set-upstream --force origin phase2
\end{verbatim}

\section{Příprava na vydání}
\subsection{F*** bomba}

Byl použit příkaz:

\begin{verbatim}
git filter-branch --tree-filter "git grep -l 'fuck' | xargs
sed -i '' -e 's/TODO.*fuck.*$/TODO/g'" -f --tag-name-filter cat -- --all HEAD
\end{verbatim}

který jsem už pomocí předchozích příkazů a znalostí z PS1 vymyslel.

\subsection{Smazání API klíčů}
Byly použity příkazy:
\begin{verbatim}
cp js/keys.js js/keys.js.backup
git filter-branch --force --tree-filter '
git rm -f --ignore-unmatch js/keys.js' --tag-name-filter cat -- --all HEAD
mv js/keys.js.backup js/keys.js
\end{verbatim}

které nejdříve skopírují soubor \texttt{js/keys.js}, následně smaže tento soubor že všech commitů a uloží soubor lokálně.

\subsection{Oštítkování verze 0.1.0}

Byl použit příkaz:

\begin{verbatim}
git log --patch -- VERSION
\end{verbatim}

který vrátil adresu commitu, kde byl změněn soubor z verze \texttt{0.0.1\_testing} na \texttt{0.1.0}.

\begin{verbatim}
git checkout 6cdf14c9850fa49a6a27fea90853e41a45a54da8
git tag v0.1.0
git checkout master
\end{verbatim}

tag a checkout znám z přednášek.

\subsection{Vývojářská větev}

Byly použity příkazy:

\begin{verbatim}
git branch dev
git reset --hard v0.1.0
git checkout dev
git push --force origin dev
git push --force origin dev --tags
git push --force origin master
git push --force origin master --tags
\end{verbatim}

Pro jistotu jsem udělal push ještě s přepínačem \texttt{--tags}.

Inspirace ze \href{https://stackoverflow.com/questions/1628563/move-the-most-recent-commits-to-a-new-branch-with-git}{stackoverflow}.

\subsection{Vývojářská větev}
Byly použity příkazy:

\begin{verbatim}
git branch dev
git reset --hard v0.1.0
git checkout dev
\end{verbatim}

\subsection{Nasazení hlavních větví repozitáře}
Byly použity příkazy:
\begin{verbatim}
git push --force origin dev
git push --force origin dev --tags
git push --force origin master
git push --force origin master --tags
\end{verbatim}

Pro jistotu jsem udělal push ještě s přepínačem \texttt{--tags}.

Inspirace ze \href{https://stackoverflow.com/questions/1628563/move-the-most-recent-commits-to-a-new-branch-with-git}{stackoverflow}.

\section{Vychytávky}

\subsection{Špatné konce řádků}

Jeden ze způsobů, jak to vyřešit je příkaz \texttt{git config}.

Postup se liší podle toho, na kterém OS se nacházíme:

\begin{enumerate}
    \item Na \emph{Linux/OSX}, provedeme příkaz:
        \begin{verbatim}git config --global core.autocrlf input\end{verbatim}
        Tento příkaz nám převede jakékoliv \texttt{CRLF} na \texttt{LF} při \texttt{commitu}.
    \item Na \emph{Windows}, provedeme příkaz:
        \begin{verbatim}git config --global core.autocrlf true\end{verbatim}
        Tento příkaz nám při \texttt{checkoutu} převede \texttt{LF} na \texttt{CRLF} a při \texttt{commitu} zpátky \texttt{CRLF} na \texttt{LF}.
\end{enumerate}

Tyto příkazy jsem zjistil ze \href{https://stackoverflow.com/questions/10418975/how-to-change-line-ending-settings?utm_medium=organic&utm_source=google_rich_qa&utm_campaign=google_rich_qa}{stackoverflow}.


\subsection{Výpis TODO při commitu}

Toto bych řešil pomocí \texttt{git hook}.

Nejdřív je za potřebí soubor, do kterého zadáme script, který se bude volat po každém commitu.

\begin{verbatim}
vim .git/hooks/post-commit
\end{verbatim}

Poté do něj vložíme nějaký script. Nejspíše by stačilo mít nejaký soubor, ve kterém budou vývojáři při každém commitu připisovat nějaké TODO.

\begin{verbatim}
#!/bin/bash
vim TODO.txt
git commit -m "update TODO list"
\end{verbatim}

Na závěr už jen stačí soubor post-commit udělat executable:

\begin{verbatim}
chmod +x .git/hooks/post-commit
\end{verbatim}

Inspirace: \href{https://www.digitalocean.com/community/tutorials/how-to-use-git-hooks-to-automate-development-and-deployment-tasks}{zde}.

\end{document}
